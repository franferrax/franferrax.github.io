\documentclass[12pt, a4paper]{article}
\usepackage[margin=2cm, top=2.5cm]{geometry}
\usepackage[english]{babel}
\usepackage{fouriernc}
\usepackage[T1]{fontenc}
\usepackage[parfill]{parskip}
\usepackage{tabularx}
\usepackage{graphicx}
\usepackage{xcolor}
\usepackage{fancyhdr}
\usepackage{multirow}
\usepackage{titlesec}
\usepackage[fixed]{fontawesome5}
\usepackage[hidelinks]{hyperref}
\usepackage{bold-extra}

% Colors
\definecolor{BackGround}      {HTML} {EDE7D9}
\definecolor{SectionTitle}    {HTML} {D1603D}
\definecolor{SubSectionTitle} {HTML} {688B58}
\definecolor{ForeGround}      {HTML} {4B4237}
\definecolor{Links}           {HTML} {247BA0}

% Page style
\fancypagestyle{MyPageStyle}{%
    \setlength{\headheight}{0em}%
    \fancyhead{}%
    \fancyfoot{}%
    \renewcommand{\headrule}{}%
    \rfoot{\textcolor{SectionTitle}{\thepage}}%
}%
\pagestyle{MyPageStyle}
\pagecolor{BackGround}

% Sections style
\titleformat{\section}
{\bfseries \Large \color{SectionTitle}}{\thetitle.}{0.5em}
{\vspace{0.8em} \hrule height 0.15em}
% Subsections style
\titleformat{\subsection}
{\bfseries \large \color{SubSectionTitle}}{\thetitle.}{0.5em}
{\vspace{0.4em} \hrule height 0.05em}
% Subsubsection style
\titleformat{\subsubsection}
{\bfseries \normalsize \color{SubSectionTitle}}{\thetitle.}{0.5em}
{}

% Commands and document metadata
\newcommand{\name}{Francisco Ferrari Bihurriet}
\newcommand{\location}{Castelldefels, Catalonia, Spain}
\newcommand{\linkedin}{/in/francisco-ferrari-bihurriet}

\newcommand{\addTitle}[1]{\Huge\textbf{#1}\vspace{0.3em}}
\newcommand{\addDetail}[2]{\faIcon[regular]{#1} & \large{#2} & \\}

\newcommand{\mysep}{{\Large $\mid$}\ }
\newcommand{\titledItem}[2]{\item \textbf{#1:} #2}
\newcommand{\myhref}[2]{\textcolor{Links}{\href{#1}{#2}}}
\newcommand{\myPic}[1]{{%
    \setlength{\fboxsep}{0em}%
    \setlength{\fboxrule}{0.08em}%
    \fbox{\includegraphics[width=#1, keepaspectratio]{me}}%
}}%

\newcommand{\lib}[1]{\texttt{\textbf{#1}}}
\newcommand{\python}{\emph{Python}}
\newcommand{\cc}{\emph{C}}
\newcommand{\cpp}{\emph{C++}}

\title{\name}
\author{\name}
\begin{document}
    \thispagestyle{MyPageStyle}
    \color{ForeGround}

    % Personal Details
    \hrule height 0.3em \vspace{0.5em}
    \begin{center}
        \begin{tabular}{rlr}
            \multicolumn{2}{l}{\addTitle{\name}} & \multirow{3}{*}{\myPic{2.6cm}} \\
            \vspace{0.3em}
            \addDetail{map}{\location}
            \addDetail{linkedin-in}{\myhref{https://www.linkedin.com\linkedin}{\linkedin}}
            \vspace{0.3em}
        \end{tabular}
    \end{center}


    \section*{About} \label{sec:about}
    I am a senior engineer with more than 10 years of experience in software development and application security.
    For 7 years, I held a technical role in the development of \emph{Core Impact},
    a leading penetration testing and offensive security product by \emph{Core Security} (now \emph{Fortra}).
    In 2022, I joined \emph{Red Hat} to work in the security team of the \emph{OpenJDK} project.
    As a professional, I am interested in low-level programing, compilers,
    network protocols, cryptography, and security vulnerabilities,
    including analysis and exploitation for testing purposes.

    \section*{Skills} \label{sec:skills}
    \begin{itemize}
        \titledItem{Seek for excellence}{%
            detail oriented person with a spirit of continuous improvement,
            passionate, motivated and persevering
        }
        \titledItem{Communication}{%
            friendly, engaged and accountable,
            responsive in chats or email,
            glad to take any opportunity to improve my presentation skills
        }
        \titledItem{Collaboration}{%
            familiar with pair programming,
            experience interacting with other teams to coordinate release cycles,
            internal demos and product integrations
        }
        \titledItem{Planning}{%
            autonomous assignment of tasks,
            ability to organize and prioritize my backlog,
            identifying intermediate milestones and reporting progress
        }
        \titledItem{Analytical skills}{%
            exposure to various bug fixes through root-cause analysis,
            combining both static and dynamic analysis with
            creative ways to minimize the steps to reproduce
        }
        \titledItem{Value of automation and documentation}{%
            proactive approach towards automating repetitive tasks and simplifying workflows,
            writing clarity when documenting tasks or redacting defect reports,
            recurring to diagrams or graphics when necessary
        }
        \titledItem{Security Mindset / Reverse Engineering}{%
            basic usage of \emph{IDA Pro} / \emph{WinDBG} / \emph{gdb},
            involvement in some \emph{capture-the-flag} contests,
            reversing and exploiting toy applications
        }
    \end{itemize}

    %\pagebreak
    \section*{Work Experience} \label{sec:experience}

    \subsection*{Red Hat \mysep Argentina/Spain \mysep 2022 --- Present} \label{subsec:job1}
    I have been working at \emph{Red Hat} for more than 3 years,
    contributing to several security libraries enhancements,
    and participating of the quarterly \emph{Critical Patch Updates (CPU)}.

    \subsubsection*{Sr Software Engineer (Red Hat) \mysep Jan 2022 --- Present}
    \begin{itemize}
        \item Working for a sub-team of the \emph{OpenJDK} runtime team,
        we have been in charge of the FIPS-mode integration
        for the \emph{OpenJDK} RPM packages in RHEL
        \begin{itemize}
            \item This has been developed as a set of downstream patches,
            combined with upstream enhancements and fixes
            \item Through my upstream contributions,
            I became a committer of the \emph{OpenJDK} upstream project
            \item I also interact in specific customer cases
            related to \emph{OpenJDK} security,
            when the support engineers demand our assistance
        \end{itemize}
        \item As a member of the security sub-team, I also participated in
        recent \emph{Critical Patch Updates (CPU)}, including:
        \begin{itemize}
            \item Analysis of security vulnerabilities in any part of the runtime,
            including machine code generated by the JIT compilers (C1 and C2)
            \item Development of security reproducers triggering the vulnerability,
            or improvements to the already available reproducers
            \item Review of the proposed fix by \emph{Oracle}, with special attention to
            possibly unsolved variants of the original bug
            \item Backports for the long-term releases maintained by the community
            (as of 2025, these include \emph{OpenJDK} 8, 11, 17 and 21)
            \item Participation in disussions within the \emph{OpenJDK Vulnerability Group (VG)},
            a multi-company group where most important vendors have representation
            \item Reviews and testing of backports developed by other members of the VG
        \end{itemize}
        \item I moved to Spain in Mar 2023, keeping my current position
    \end{itemize}

    \subsection*{HelpSystems (ex Core Security) \mysep Argentina \mysep 2015 --- 2021} \label{subsec:job1}
    I worked at this company for almost 7 years, achieving 3 promotions.
    Our team shipped 11 releases of \emph{Core Impact}.

    \subsubsection*{Sr Developer (Core Security - Help Systems) \mysep Jun 2019 --- Dec 2021}
    \begin{itemize}
        \item Being the most experienced engineer in the team, I usually
        provided guidance to new hires with the onboarding and ramp-up process
        \item Holding a technical leadership role, I became
        a dependable point of reference for other members and teams
        when they needed to discuss architectural changes
        or dive into the implementation details of some areas of the product
        \item Comprehending the importance of the product quality and stability,
        I was also involved in maintaining the product testing framework
        and its associated testing lab composed by thousands of virtual machines
        \item I was very active in the team chat
        and often happy to answer any question,
        I also took an important role in the meetings
        when discussing how to address new requirements
        \item In late 2020 / early 2021: led a major upgrade of
        our modified version of the \lib{OpenSSL} library (v1.0.1 $\rightarrow$ v1.1.1),
        from the planning phase to the final integration
        \begin{itemize}
            \item This upgrade also extended to our modified version of the \lib{pyOpenSSL} \python{} bindings,
            whose upstream project had evolved from a \emph{CPython} extension to a pure-\python{} library
            relying on the \lib{cryptography} library \lib{cffi} bindings
            (that implied porting our patches to expose the extended APIs by our \lib{OpenSSL} modifications)
            \item As part of this work, several \python{} libraries ended up integrated
            into our codebase, requiring changes in our non-standard \python{} build system
            (due to the way in which the interpreter is customized and embedded in the product)
            \item In the end, we finished with a renewed cryptography stack,
            removing legacy bindings and with a simpler mechanism to integrate third party libraries
        \end{itemize}
        \item In 2019: actively worked in the refinement of the requirements
        for a highly prioritized update of our licensing mechanism,
        coordinating infrastructure changes across multiple teams
        \begin{itemize}
            \item Helped to envision the shortest path to achieve our goal,
            and played a key role conveying that idea
            and convincing the team that it would work
            \item As part of the tasks, took responsibility for modifying
            a very legacy copy protection mechanism written in (old) \cpp{},
            which represented the main source of uncertainty
            regarding the feasibility of the requirement
        \end{itemize}
    \end{itemize}

    \subsubsection*{Ssr Developer (Core Security) \mysep May 2017 --- May 2019}
    \begin{itemize}
        \item Rewrote a \emph{phishing} webpage cloning module to change its approach,
        by mounting a \emph{man-in-the-middle reverse proxy}
        (also took part in the previous design/analysis)
        \item Worked in pairs for an urgent security patch
        to the updates channel of our product,
        dealing with completely unknown code
        and a legacy unmaintained cloud infrastructure;
        we refactored \python{} code to remove duplicated logic,
        added tests and shipped the fix in the expected time
        \item Started to work in multiple parts of the product,
        fixing minor issues in components that I was less used to
        (such as \emph{JavaScript} graphical user interface code),
        and developing tests which required specific \emph{Linux} server configurations
    \end{itemize}

    \subsubsection*{Jr Developer (Core Security) \mysep Jan 2016 --- Apr 2017}
    \begin{itemize}
        \item Collaborated extending a generic mechanism
        to remotely execute library functions on \emph{Linux} agents,
        involving \emph{dynamically linked libraries}, function resolving mechanisms,
        the \emph{System V Application Binary Interface} and its \emph{calling conventions}
        \begin{itemize}
            \item This task required writing \emph{x86-64 inline assembly} in \cc{}
            and assembled instructions as raw bytes in \python{},
            as well as remotely debugging our agent without symbols
            (since it basically consists of \emph{position independent} stripped \cc{} code)
        \end{itemize}
        \item Performed medium-scale \cc{} and \python{} refactors to reorganize some APIs
        and reduce boilerplate code
    \end{itemize}

    \subsubsection*{Trainee/Intern Developer (Core Security) \mysep Jan 2015 --- Dec 2015}
    \begin{itemize}
        \item Collaborated extending the post-exploitation infrastructure of \emph{Android} agents (in \cc{})
        \item Implemented \cc{} structures member alignment through padding bytes,
        in an in-house \python{} library interfacing with low-level APIs
        (also wrote the corresponding test suite, which included a testing \emph{dynamically linked library})
    \end{itemize}


    \section*{Education} \label{sec:education}

    \subsection*{Electronics Engineer \mysep Universidad de Buenos Aires} \label{subsec:edu1}
    I started the \emph{Electronics Engineer} career in 2010.
    I began my first work internship in 2015 (recommended by one of my programming teachers),
    and realized that my professional career would evolve towards the information security area.
    I originally decided to keep on \emph{Electronics Engineer}
    since I like low-level programming, information security, and enjoy learning about hardware.
    Over time I continuosly reduced the academic work to focus on my professional development,
    and after making some progress in background for a while,
    I finally left at approximately 50\% of the career.
    \begin{itemize}
        \titledItem{Overall grade-point average}{$7,31$}
    \end{itemize}

    \subsection*{Exploit Writing Trainings \mysep Core Security \mysep 2016 \& 2019} \label{subsec:edu2}
    \begin{itemize}
        \titledItem{Introduction to Windows Kernel Exploitation}{%
            two days internal training from the exploit writing team,
            including exercises with intentionally vulnerable drivers
        }
        \titledItem{Introduction to Exploits Development}{%
            two days hands-on training from the exploit writing team,
            covering topics like \emph{buffer overflows}, \emph{integer overflows} and \emph{use after free}
        }
    \end{itemize}

\end{document}
